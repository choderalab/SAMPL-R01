%%%%%%%%%%%%%%%%%%%%%%%%%%%%%%%%%%%%%%%%%%%%%%%%%%%%%%%%%%%%%%%%%%%%%%%%%%%
% * <sonya.hanson@choderalab.org> 2015-04-23T18:56:00.248Z:
%
% 
%
% HEADER: DON'T EDIT THIS!
%%%%%%%%%%%%%%%%%%%%%%%%%%%%%%%%%%%%%%%%%%%%%%%%%%%%%%%%%%%%%%%%%%%%%%%%%%%
% ^ <sonya.hanson@choderalab.org> 2015-04-23T18:57:46.938Z.

\documentclass[11pt]{article}

\title{ Advancing predictive physical modeling through focused development of model systems to drive new modeling innovations}

\usepackage[top=0.5in, bottom=0.5in, left=0.5in, right=0.5in]{geometry}
\usepackage{helvet}
\usepackage{url} % hypderref?
\usepackage{graphicx}
\graphicspath{{figures/}} % The figures are in a figures/ subdirectory.
\renewcommand{\familydefault}{\sfdefault}
\pagestyle{empty}
%\pagestyle{plain}

% Fancy page-width tables
\usepackage{tabularx}

% Use a package for framed boxes
\usepackage{mdframed}

\usepackage[T1]{fontenc}
\usepackage{amssymb}


\usepackage{setspace}
\usepackage{microtype}

\usepackage{amsfonts}
\usepackage{amsmath}

\usepackage{floatrow}

\usepackage[normalem]{ulem} % for nci.bst

\usepackage{sidecap}
\usepackage[abs]{overpic}
\usepackage{wrapfig}

%\usepackage[round,authoryear]{natbib}
\usepackage{cite}
%\setlength{\bibsep}{0.00in}

\usepackage{hyperref}
\hypersetup{colorlinks=true, urlcolor=black, citecolor=black, linkcolor=black}

\newcommand{\doi}[1]{\href{http://dx.doi.org/#1}{doi:#1}}

\newcommand{\ac}[1]{{\sc \lowercase{#1}}}

\renewcommand{\baselinestretch}{.93}
%\renewcommand{\baselinestretch}{.90}
\usepackage{wrapfig} 

\usepackage{bibspacing}
\setlength{\bibspacing}{\baselineskip}


\graphicspath{{figs/}}

\makeatletter

\newcommand{\captionfonts}{\footnotesize}

\makeatletter  % Allow the use of @ in command names
\long\def\@makecaption#1#2{%
  \vskip\abovecaptionskip
  \sbox\@tempboxa{{\captionfonts #1: #2}}%
  \ifdim \wd\@tempboxa >\hsize
    {\captionfonts #1. #2\par}
  \else
    \hbox to\hsize{\hfil\box\@tempboxa\hfil}%
  \fi
  \vskip\belowcaptionskip}      
\makeatother

\renewcommand{\figurename}{{\bf Figure}}

% Page numbering.
%\pagestyle{plain}
%\pagenumbering{arabic}

\setlength{\abovecaptionskip}{-5pt}

\makeatother

\renewcommand{\refname}{Bibliography and References Cited}

\setlength{\parindent}{0pt} % Don't indent first line
%\setlength{\parskip}{1ex plus 0.5ex minus 0.2ex} % Add some space between paragraphs
\setlength{\parskip}{0.8ex} % Add some space between paragraphs

\begin{document}

%======================================
% CITE OUR REFS FIRST
%\phantom{
%\cite{mobley-chodera-dill:2006:jcp:orientation-restraints,mobley-chodera-dill:2007:jctc:confine-and-release,shirts-mobley-chodera-pande:2007:jpcb:dispersion-corrections,chodera:jcp:2007,shirts-mobley-chodera:2007:annu-rep-comput-chem:prime-time,shirts-chodera:jcp:2008:mbar,ncmc,chodera-shirts:jcp:2011:gibbs,chodera:curr-opin-struct-biol:2011:drug-discovery,chodera-shirts:jcamd:2013:yank}
%\cite{gunner:biophys-j:1997:mcce,gunner:bba:2000:proton-electron-transfer,gunner:biophys-j:2002:mcce,gunner:j-comput-chem:2009:mcce2,gunner:jmb:2009:mcce2-hsa,gunner:proteins:2010:reaction-center,dutton:biochem:1994:photosynthetic-reaction-center,gunner:proteins:2010:reaction-center,gunner:photosynth-res:2013:photosynthetic-reaction-center,gunner:bba:2000:proton-electron-transfer,gunner:j-comput-chem:2009:mcce2,nielsen-gunner-garciamoreno:proteins:2011:pka-cooperative,stanton-houk:jctc:2008:benchmarking-pka-prediction}
%}

%%%%%%%%%%%%%%%%%%%%%%%%%%%%%%%%%%%%%%%%%%%%%%%%%%%%%%%%%%%%%%%%%%%%%%%%%%%
% SPECIFIC AIMS
%%%%%%%%%%%%%%%%%%%%%%%%%%%%%%%%%%%%%%%%%%%%%%%%%%%%%%%%%%%%%%%%%%%%%%%%%%%

%{\large \bf SPECIFIC AIMS}

%\eject

%%%%%%%%%%%%%%%%%%%%%%%%%%%%%%%%%%%%%%%%%%%%%%%%%%%%%%%%%%%%%%%%%%%%%%%%%%%
% SIGNIFICANCE
%%%%%%%%%%%%%%%%%%%%%%%%%%%%%%%%%%%%%%%%%%%%%%%%%%%%%%%%%%%%%%%%%%%%%%%%%%%

{\large \bf SIGNIFICANCE}
%2 pages

%Promise of physical methods, how they could transform drug discovery and the design of new small molecules for chemical biology
% (If space permits here we will allude also to how this could help with broader areas too such as engineering of biologics)
Physical methods are poised to transform drug discovery and chemical biology via quantitative, predictive design of small molecules. %Single sentence placeholder for outline

%Limitations of methods and of retrospective tests or application for methods improvements
Unfortunately, these methods still have severe limitations and much work is needed to expand their domain of applicability, but retrospective tests are unsuitable for this. %Single sentence placeholder for outline

%Explain need for blind challenges and how D3R only fills part of need
To truly advance these methods, we need a series of blind challenges focused on pushing the limits of predictive techniques. %Single sentence placeholder for outline

%Discuss importance of SAMPL (including publication/citation count, previous successes in advancing the field, focus), need for funding to:
%   a. Continue SAMPL
%   b. Extend SAMPL to bridge to D3R rather than leaving a ?capability gap? (here?s where I explain why this is distinct from D3R -- this is crucial)
SAMPL has been vital to fulfill this role; here we propose to continue and extend SAMPL via collection of a set of carefully selected experimental data in order to drive further improvements in modeling. %Single sentence placeholder for outline

%Wrap up with how this will realistically help advance binding prediction in a realistic timeframe (a brief mention of CASP will likely be appropriate even though we don?t follow the same model)
This work is necessary in order to advance modeling to the point where it can reliably guide drug discovery efforts, reducing time consuming and costly trial and error. %Single sentence placeholder for outline


{\large \bf INNOVATION}
% 1 page

%SAMPL drives science in our groups and in the field
   %a. Give examples from our groups from prior SAMPLs
   %b. Examples from field in prior SAMPLs
         %b.1 Figure showing progression in hydration free energy accuracy across a couple of SAMPLs or comparing two SAMPLs?
    %c. Mention uniqueness of SAMPL -- there are other predictive challenges (D3R, pKa coop, CASP) but none focused on driving quantitatively predictive protein-ligand modeling
We will discuss how SAMPL has previously driven innovation in the field.


% Innovation in experimental pipelines (Aim 3)
Innovation here will include developing new, high-throughput experiments for studying protein ligand binding (Aim 3).

% Innovative reference calculations (Aim 4)
In Aim 4, we will not only run SAMPL community challenges, but also perform our own reference calculations with the latest techniques, testing their accuracy and using these to assess the current state-of-the-art.

% Innovation in analysis of method performance, comparison of methods, workflow science? (Aim 4)
Innovation here will also involve careful assessment of how to compare methods and analyze their relative performance in a statistically sound way.

{\large \bf APPROACH}
% 9 pages


%Aim 1 - Generate new data for simple SAMPL blind challenges on physical property prediction (1.5 pages)
{\bf Aim 1: Generate new data for ``simple'' SAMPL blind challenges on physical property prediction.}
We will develop new solution-phase datasets for druglike small molecules. These data can test critical aspects of small molecule modeling (including accounting for interactions and treatment of protonation/tautomeric state) and improve our ability to predict physical properties relevant to drug discovery in new regions of chemical space. We will initially focus on distribution between organic phases and on pKa?s and their modulation by solvent environment, using these data to drive improvements in the modeling of ligand interactions.

%Explain what science -- log D/logP, pKa primarily but mention possible other areas of interest
    %Industry partnerships (and past precedent)
%Compare log D accuracy vs hydration accuracy, highlight issues and explain relevance to drug discovery; note lessons learned in SAMPL5



%Aim 2 - Measure binding of novel host-guest complexes for introductory ligand binding challenges (2 pages)
{\bf Aim 2: Measure binding of novel host-guest complexes for introductory ligand binding challenges.}
We will measure new host-guest binding free energies for cucurbiturils and deep-cavity cavitands, yielding further host-guest binding challenges which span between physical property prediction and protein-ligand binding. Host guest systems are some of the simplest cases of molecular recognition, and thus these binding data will drive improvements in modeling of simple binding systems with techniques of relevance to drug discovery.

%Intro lessons learned on HG systems, relevance to drug discovery
%Isaacs science
%Gibb science


%Aim 3 - Generate biologically relevant advanced model systems for protein-ligand binding challenges. (2 pages) 
{\bf Aim 3. Generate biologically relevant advanced model systems for protein-ligand binding challenges.}
We will identify suitable biological protein-ligand model systems (difficult but tractable in order to push the limits of physical techniques) then measure binding and develop these for blind challenges. This will include binding studies on human serum albumin and bromodomains or aspartyl proteases; initial binding data will be expanded by the selection of additional ligands or the creation of mutations in the protein that modulate binding.


%Aim 4 - Coordinate, run, and analyze blind challenges to advance modeling of binding (1.5 pages)
{\bf Aim 4. Coordinate, run, and analyze blind challenges to advance modeling of binding.}
The data collected in Aims 1-3 will drive annual SAMPL blind challenges, allowing the field to test the latest methods and force fields to assess progress, compare them against one another head-to-head, and perform sensitivity analysis to learn how much different factors (protonation state, tautomer selection, solvent model, force field, sampling method, etc.) affect predictive power. Results will then feed back into improved treatment of these factors for subsequent challenges, driving regular cycles of application, learning, and advancement.
%Coordinate and run SAMPL blind challenges
	%Run reference calculations to:
		%a. Test current standard methods/FF
		%b. Facilitate others learning (swap method or FF)
		%c. Do sensitivity analysis (learn what?s important)
		%d. (Give examples of what we?ve learned from this)
		%e. (We will make inputs, outputs, and methods available too)
	%Select and announce null models, run them
	%Do statistical analysis of results (& compare to nulls), report back
	%Work with D3R on meeting coordination
	%Coordinate follow-up experiments as needed (cite examples when this was desirable)
	%Coordinate with JCAMD on special issues
	%Data archival and dissemination
	%D3R coordination:
	%	a. Co-running workshops with D3R
	%     b. Coordinating challenges with them, submission deadlines offset from D3R challenges



%%%%%%%%%%%%%%%%%%%%%%%%%%%%%%%%%%%%%%%%%%%%%%%%%%%%%%%%%%%%%%%%%%%%%%%%%%%%%%%%%%%%%%%%%%%%%%%%%%%%%%
% TIMELINE
%%%%%%%%%%%%%%%%%%%%%%%%%%%%%%%%%%%%%%%%%%%%%%%%%%%%%%%%%%%%%%%%%%%%%%%%%%%%%%%%%%%%%%%%%%%%%%%%%%%%%%

{\bf \large TIMELINE} %1 page minus a paragraph
%"The timeline is actually likely to be very important here, so I'd strongly suggest we include it, if not expand it to one full page. We have to "sell" the reviewers on how the concept would play out into actual challenges, advances, etc. so that they get a concrete idea of all the good things that will come of this. We should mention when we would hold blind challenges, when data would be released, when meetings would occur, when benchmarks would be published, and what datasets would be generated for the community."



%%%%%%%%%%%%%%%%%%%%%%%%%%%%%%%%%%%%%%%%%%%%%%%%%%%%%%%%%%%%%%%%%%%%%%%%%%%%%%%%
% FIGURE: AIMS OVERVIEW
%%%%%%%%%%%%%%%%%%%%%%%%%%%%%%%%%%%%%%%%%%%%%%%%%%%%%%%%%%%%%%%%%%%%%%%%%%%%%%%%
%\begin{figure}[h]
%\begin{centering}
%\resizebox{\textwidth}{!}{\includegraphics{figures/timeline.pdf}}

%\end{centering}
%\vspace{0.1in}
%\caption{\footnotesize {\bf Timeline.}
%\label{figure:aims-overview}}
%\end{figure}
%%%%%%%%%%%%%%%%%%%%%%%%%%%%%%%%%%%%%%%%%%%%%%%%%%%%%%%%%%%%%%%%%%%%%%%%%%%%%%%%

%%%%%%%%%%%%%%%%%%%%%%%%%%%%%%%%%%%%%%%%%%%%%%%%%%%%%%%%%%%%%%%%%%%%%%%%%%%%%%%%%%%%%%%%%%%%%%%%%%%%%%
% COLLABORATION MANAGEMENT PLAN
%%%%%%%%%%%%%%%%%%%%%%%%%%%%%%%%%%%%%%%%%%%%%%%%%%%%%%%%%%%%%%%%%%%%%%%%%%%%%%%%%%%%%%%%%%%%%%%%%%%%%%

{\large \bf COLLABORATION MANAGEMENT PLAN} % 1 paragraph


{\large \bf OUTLOOK} %or conclusions. 0.5 page


%%%%%%%%%%%%%%%%%%%%%%%%%%%%%%%%%%%%%%%%%%%%%%%%%%%%%%%%%%%%%%%%%%%%%%%%%%%%%%%%%%%%%%%%%%%%%%%%%%%%%%
% BIBLIOGRAPHY
%%%%%%%%%%%%%%%%%%%%%%%%%%%%%%%%%%%%%%%%%%%%%%%%%%%%%%%%%%%%%%%%%%%%%%%%%%%%%%%%%%%%%%%%%%%%%%%%%%%%%%

\eject

%\footnotesize
%\scriptsize
%\bibliographystyle{acm}
\bibliographystyle{nci}
%\bibliographystyle{nar}
\bibliography{sampl-r01}

\end{document}